%% 美赛模板:正文部分

\documentclass[12pt]{article}  % 官方要求字号不小于 12 号,此处选择 12 号字体

% 本模板不需要填写年份,以当前电脑时间自动生成
% 请在以下的方括号中填写队伍控制号
\usepackage[2417377]{easymcm}  % 载入 EasyMCM 模板文件
\problem{A}  % 请在此处填写题号
%\usepackage{mathptmx}  % 这是 Times 字体,中规中矩 
\usepackage{mathpazo}  % 这是 COMAP 官方杂志采用的更好看的 Palatino 字体,可替代以上的 mathptmx 宏包

\usepackage{threeparttable}

\title{An MCM Paper Made by Team 2417377 Based on}  % 标题

% 如需要修改题头(默认为 MCM/ICM),请使用以下命令(此处修改为 MCM)
%\renewcommand{\contest}{MCM}

%此处用于给公式编号
\makeatletter
\@addtoreset{equation}{section}
\makeatother
\renewcommand{\theequation}{\arabic{section}-\arabic{equation}}


% 文档开始
\begin{document}

% 此处填写摘要内容
\begin{abstract}
    As the...//Considering...(background)

    For problem 1, we

    For problem 2, we 

    For problem 3, we
    
    Eventually,  

    % 美赛论文中无需注明关键字。若您一定要使用,
    % 请将以下两行的注释号 '%' 去除,以使其生效
    \vspace{5pt}
    \textbf{Keywords}: Lotka–Volterra, mathematics, \LaTeX.

\end{abstract}

\maketitle  % 生成 Summary Sheet
\tableofcontents  % 生成目录


% 正文开始
\section{Introduction}
\subsection{Problem Background}
A population that is necessarily dependent on reproduction for its development and continuation. Based on sexual reproduction, the growth rate of a population is greatly influenced by the sex ratio. Therefore, although many species exhibit a 1:1 sex ratio at birth, species tend to deviate from an even sex ratio in order to adapt to their environment and continue their populations. This is known as adaptive sex ratio variation.\cite{1} The lamprey is one such typical species.

However, the role of the lamprey is complex. the seven-gill eel is complex. For some lake habitats, it is a parasite that is harmful to the ecosystem, and we want to reduce its reproduction; but at the same time, the seven-gill eel is a source of food in some parts of the world, and we want to promote its reproduction.\cite{2} Therefore, we address these questions by studying the pros and cons of the ability to alter their sex ratio based on resource availability, modeling their relationship with growth rates, and studying their impact on other species in the ecosystem

\begin{figure}[htbp]
	\centering
	\includegraphics[width=.6\textwidth]{a fish with lamprey.jpg}
	\caption{a fish with lamprey}\label{fig:001}
\end{figure}

\subsection{Restatement of the Problems}

\begin{itemize}
	\item Develop and examine a model to provide insights into the impact on the larger ecological system when the population of lampreys can alter its sex ratio
	\item Evaluate the advantages and disadvantages of the ability to change sex ratios for the population itself and for the external ecosystem under the same resource availability conditions, taking into account the modeling of Question 1
	\item Study the impact of sex ratio on ecosystem stability, specifically speaking,to evaluate the impact on ecosystems by considering the influence of changing population sex ratios on other populations on the basis of previous models.
	\item Developing a model to describe the relationship between lampreys and its parasites, study whether ecosystems with changing population sex ratios can provide advantages to other species in the ecosystem such as parasites
\end{itemize}

\subsection{Overview of Our work}
In order to find the relationship between sex ratio and environmental availability, we modeled the dynamics of the population and gradually added predator and parasite interactions to the model, constituting a model of the effect of variable sex ratio on the ecosystem, and finally, based on the model, we proposed the role of sex ratio control for human beings in the ecosystem.


\begin{itemize}
	\item Assumptions are made to reduce complex ecosystems and the interactions within them (e.g., food chain relationships) to plausible resource-pressure relationships, and typical predation and parasitism relationships are included for objective modeling of population changes.
	\item Modeling of populations. Based on the Logitics model and the population dynamics equation, a sex ratio-environmental availability model was developed to study its effects on the population itself, as well as on the ecosystem.
	\item Modeling ecosystem interactions. Based on the Lotka-Volterra equation, predators and parasitoids were introduced respectively, and the model was improved so that it could describe the interactions of various groups in the ecosystem to study the role of sex ratios, and the TOPSIS evaluation model improved by the entropy weight method was used to study the effects on ecosystem stability
	\item Analyze the strengths and weaknesses of the model and make the conclusions
\end{itemize}



\begin{figure}[htbp]
	\centering
	\includegraphics[width=0.95\textwidth]{work.jpg}
	\caption{our work}\label{fig:work}
\end{figure}


\section{Assumptions and Justification}
 While ensuring the correctness,to simplify the problem, we make the following basic assumptions, each of which is properly justified.

\begin{itemize}
	\item \textbf{Assumption 1:}
	
	$\hookrightarrow$Justification:...\cite{1}
	\item \textbf{Assumption 2:}
	
	$\hookrightarrow$Justification:...\cite{2}
	\item \textbf{Assumption 3:}
	
	$\hookrightarrow$Justification:...\cite{3}
\end{itemize}


\section{Notations}
The primary notations used in this paper are listed in Table.

% 三线表示例
\begin{table}[!htbp]
\begin{center}
\begin{threeparttable}
\caption{Notations}
\begin{tabular}{cl}
	\toprule
	\multicolumn{1}{m{3cm}}{\centering Symbol}
	&\multicolumn{1}{m{12cm}}{\centering Definition}\\
	\midrule
	$N_{F}$&the number of female lamprey\\
	$N_{M}$&the number of male lamprey\\
	$TNF$ &total number of lamprey\\
	$RA$&resource availability\\
	$c$&Population fertility coefficient\\
	$r$ &the growth rate relative to sex ratio\\
	$\alpha$ &the proportion of male in growth lamprey\\
	$\beta$ &the decrease rate in relation to number\\
	$U$&the number of hosts and predators\\
	$h$&the growth rates of hosts and predators\\
	$\epsilon$&the predation conversion factor\\
	$a$&predatory capacity of lamprey\\
	\bottomrule
\end{tabular}\label{tb:001}
\small
\textit{Other notations instructions will be given in the text.}
\end{threeparttable}
\end{center}
\end{table}

\section{The Models and The solution}

\subsection{Model 1 and Solution}
\subsubsection{Details about Model 1}
We define the max number of $RA$ as $RA_{max}$

\begin{equation}\label{eq:4-1}
\begin{cases}
	\frac{\mathrm{d}N_{M}}{\mathrm{d}t}=r \cdot \alpha RA-\beta N_{M}-\frac{r\cdot RA}{\lambda} \\
	\\
	\frac{\mathrm{d}N_{F}}{\mathrm{d}t}=r \cdot(1 - \alpha )RA-\beta N_{F}-\frac{r\cdot RA}{\lambda} \\
\end{cases}
\end{equation}











\begin{figure}[htbp]
	\centering
	\subfloat[]
	{\includegraphics[width=0.3\textwidth]{alpha30.jpg}\label{alpha30}}
	\quad    % 重点就在这,优先横向排列,自动换行
	\subfloat[]
	{\includegraphics[width=0.3\textwidth]{alpha40.jpg}\label{alpha40}}
	\quad
	\subfloat[]
	{\includegraphics[width=0.3\textwidth]{alpha50.jpg}\label{alpha50}}
	\quad
	\subfloat[]
	{\includegraphics[width=0.3\textwidth]{alpha60.jpg}\label{alpha60}}
	\quad
	\subfloat[]
	{\includegraphics[width=0.3\textwidth]{alpha70.jpg}\label{alpha70}}
	\quad
	\subfloat[]
	{\includegraphics[width=0.3\textwidth]{alpha80.jpg}\label{alpha80}}
	\quad
	\caption{111}
\end{figure}










\begin{table}[!htbp]
	\begin{center}
		\begin{threeparttable}
		\caption{The data of sea lamprey}
		\begin{tabular}{ccccc}
			\toprule
			\multicolumn{1}{m{2cm}}{\centering Year}
			&\multicolumn{1}{m{2cm}}{\centering \% Males}
			&\multicolumn{1}{m{2cm}}{\centering n}
			&\multicolumn{1}{m{2cm}}{\centering Males}
			&\multicolumn{1}{m{2cm}}{\centering Females}\\
			\midrule
			$2007$&53\%&3,124&1,666&1,460\\
			$2008$&57\%&2,228&1,264&964\\
			$2009$&54\%&2,725&1,485&1,240\\
			$2010$ &58\%&8,841&5,146&3,695\\
			$2011$ &60\%&10,912&6,555&4,357\\
			$2012$ &60\%&14,047&8,442&5,605\\
			$2013$ &61\%&8,947&5,495&3,452\\
			$2014$ &59\%&8,696&5,131&3,565\\
			\bottomrule
		\end{tabular}\label{tb:002}
		\small
		1. Sex ratio of sea lamprey in tributaries to Lakes Michigan and Huron\cite{3}
		
		2. the sum number of the Males is 59,522, for females the number is 59,522 
		\end{threeparttable}
	\end{center}

\end{table}










\begin{figure}[htbp]
	\centering
	\begin{subfigure}[b]{.45\textwidth}
		\includegraphics[width=\textwidth]{RA-TimeBig.png}
		\caption{Image on the left}\label{subfig:left}
	\end{subfigure}
	\begin{subfigure}[b]{.45\textwidth}
		\includegraphics[width=\textwidth]{RA-Alpha.jpg}
		\caption{Image on the right}\label{subfig:right}
	\end{subfigure}
	\caption{Two images}\label{fig:subfigures}
\end{figure}








\subsection{Model 2 and Solution}
\subsubsection{Details about Model 2}


\subsubsection{Conclusion of Model 2}
The results are shown in Figure , where $t$ denotes the time in seconds, and $c$ refers to the concentration of water in the boiler.

\begin{figure}[htbp]
\centering
\includegraphics[width=.8\textwidth]{RA-TimeBig.png}
\caption{The result of Model 2}\label{fig:666}
\end{figure}

%\clearpage
\subsubsection{Commetary on Model 2}
The instance of long and wide tables are shown in Table \ref{tb:longtable}.

% 长表格示例,更多用法请参考 longtable 宏包文档
% 以下环境及对应参数可实现表格内的自动换行与表格的自动断页
% 您也可以选择自行载入 tabularx 宏包,并通过 X 参数指定对应列自动换行
\begin{longtable}{ p{4em} p{14em} p{14em} }
\caption{Basic Information about Three Main Continents (scratched from Wikipedia)}
\label{tb:longtable}\\
\toprule
Continent & Description & Information \\
\midrule
Africa & Africa Continent is surrounded by the Mediterranean Sea to the
north, the Isthmus of Suez and the Red Sea to the northeast, the Indian
Ocean to the southeast and the Atlantic Ocean to the west. &
At about 30.3 million km$^2$ including adjacent islands, it covers 6\%
of Earth's total surface area and 20\% of its land area. With 1.3
billion people as of 2018, it accounts for about 16\% of the world's
human population. \\
\midrule
Asia & Asia is Earth's largest and most populous continent which
located primarily in the Eastern and Northern Hemispheres.
It shares the continental landmass of Eurasia with the continent
of Europe and the continental landmass of Afro-Eurasia with both
Europe and Africa. &
Asia covers an area of 44,579,000 square kilometres, about 30\%
of Earth's total land area and 8.7\% of the Earth's total surface
area. Its 4.5 billion people (as of June 2019) constitute roughly
60\% of the world's population. \\
\midrule
Europe & Europe is a continent located entirely in the Northern
Hemisphere and mostly in the Eastern Hemisphere. It comprises the
westernmost part of Eurasia and is bordered by the Arctic Ocean to
the north, the Atlantic Ocean to the west, the Mediterranean Sea to
the south, and Asia to the east. &
Europe covers about 10,180,000 km$^2$, or 2\% of the Earth's surface
(6.8\% of land area), making it the second smallest
continent. Europe had a total population of about 741 million (about
11\% of the world population) as of 2018. \\
\bottomrule
\end{longtable}

Figure \ref{fig:subfigures} gives an example of subfigures. Figure \ref{subfig:left} is on the left, and Figure \ref{subfig:right} is on the right.

% 子图(多图并列)示例,更多用法请参考 subfigure 宏包文档
% 如果您只希望几张图并列,不需要额外的 caption,那么在 figure 环境中
% 连续插入总宽度不超过 \textwidth 的多个 \includegraphics 命令即可



\section{Test the Model}

\subsection{Sensitivity Analysis}

\subsection{Error Analysis}


\section{Model Evaluation and Further Discussion}
\subsection{Strengths}
\begin{itemize}
    \item First one...
    \item Second one ...
\end{itemize}

\subsection{Weaknesses}
\begin{itemize}
    \item Only one ...
 \end{itemize}
 
 \subsection{Further Discussion}
 
 \subsubsection{Model Improvement}
 
 \subsubsection{Model Extension}


\newpage

% 参考文献,此处以 MLA 引用格式为例
\begin{thebibliography}{99}
\bibitem{1} Karlin S, Lessard S. Theoretical studies on \emph{sex ratio evolution}[J]. 1986.
\bibitem{2} Almeida P R, Arakawa H, Aronsuu K, et al. \emph{Lamprey fisheries: History, trends and management}[J]. Journal of Great Lakes Research, 2021, 47: S159-S185.
\bibitem{3} \emph{The National Center for Biotechnology Information: sea lamprey}, from\url{https://www.ncbi.nlm.nih.gov/pmc/articles/PMC5378093/}
\bibitem{4} Lewandoski, S. A. Brenden, T. O. (2022). Forecasting suppression of invasive sea lamprey in Lake Superior. Journal of Applied Ecology, 59, 2023–2035. from url{https://doi.org/10.1111/1365-2664.14203}
\end{thebibliography}



% 以下为附录内容
% 如您的论文中不需要附录,请自行删除
\begin{subappendices}  % 附录环境

\section{Appendix: Source Code on Model}
Here are the program codes we used in our research.

% 代码环境示例三则
% 如您的论文不需要展示代码,请删除
% 更多用法,请参考 listings 宏包文档

% Python 代码示例
\begin{lstlisting}[language=Python, name={test.py}]
# Python code example
for i in range(10):
    print('Hello, world!')
\end{lstlisting}

% MATLAB 代码示例
\begin{lstlisting}[language=MATLAB, name={test.m}]
% Model 1
clear;clc;
figure
alpha = 0.8;
Ram = 10000;
[t,y] = ode45(@(t,y) lampreyModel(t,y,alpha,Ram),[0,80],[800,1200]);
plot(t,y(:,1)+y(:,2),'DisplayName','Number of Lampreys','LineWidth',1);
hold on;
plot(t,y(:,1),'DisplayName','Number of Male Lampreys','LineWidth',1);
hold on;
plot(t,y(:,2),'DisplayName','Number of Female Lampreys','LineWidth',1);
hold on;
title(['Number of Lamprey vs. Time with \alpha=',num2str(alpha)],
'FontSize',16)
set(gcf,'Position',[50,50,600,500]);
xlabel('Time','FontSize',16);
ylabel('Num','FontSize',16);
legend('Location','southeast');
hold on;

% Model 1 RA-Time
clear;clc;
figure;
Ram = 10000;
for i = 0:5
alpha = i*(0.8-0.3)/5+0.3;
[t,y] = ode45(@(t,y) lampreyModel(t,y,alpha,Ram),[0,80],[800,1200]);
plot(t,Ram-y(:,1)-y(:,2),'DisplayName',
['RA with \alpha=',num2str(alpha)],'LineWidth',1.5);
hold on;
end
title('RA vs. \alpha')
legend;
hold off;

% Model 1 RA-Alpha
clear;clc;
figure;
Ram = 10000;
RaM = zeros(1000,1);
x = [(1:1000)*1/1000];
for i = 1:1000
alpha = i*1/1000;
[t,y] = ode45(@(t,y) lampreyModel(t,y,alpha,Ram),[0,200],[800,1200]);
RaM(i) = Ram-y(end,1)-y(end,2);
end
plot(x,RaM,'LineWidth',1.5,'DisplayName','RA');
hold on;
title('RA vs. \alpha')
legend;
\end{lstlisting}

% C++ 代码示例
\begin{lstlisting}[language=C++, name={test.cpp}]
// C++ code example
#include <iostream>
using namespace std;

int main() {
    for (int i = 0; i < 10; i++)
        cout << "hello, world" << endl;
    return 0;
}
\end{lstlisting}

\end{subappendices}  % 附录内容结束

\end{document}  % 结束
